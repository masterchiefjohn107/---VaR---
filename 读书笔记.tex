\documentclass[UTF8]{ctexart}
\usepackage{geometry} %页面设置
\usepackage{indentfirst} %段落缩进设置
\usepackage{color}
\usepackage{chngpage} %可以设置段落整体左右缩进等等,方法是\begin{adjustwidth}{1em}{0em}

\usepackage{iitem} %多重列表
\usepackage{float}

\usepackage{bm}

%定义一个强调格式,红色加粗,使用方法\qd{你的内容},或者\begin{qds} 你的内容 \end{qds}
\newcommand \qds {\bf \color{red}}
\newcommand \qd[1] {\begin{qds} {#1} \end{qds}}

\setCJKmainfont{微软雅黑}
\geometry{a4paper,left=1cm,right=1cm,top=2cm,bottom=2cm}
 \setlength{\parindent}{0em}
 \setlength{\parskip}{0.3em}
\title{《固定收益建模》读书笔记}
\author{john107}
\date{\today}
\begin{document}

\maketitle
\clearpage
\tableofcontents
\clearpage

\section{绪论}
一般来说,金融风险可分为市场风险、信用风险、流动性风险和操作风险,有时还有法律风险

市场风险是指金融产品市场价格和利率变化将减少银行敞口的价值,可以由风险管理者通过名义金额、敏感性和VaR衡量方法或独立监控来控制

信用风险是指交易对手信用状况的变化会影响银行敞口的封信啊,往往因为交易对手的信用等级下降或违约引起,可以通过对名义金额、当前和潜在的风险进行信用限制,如要求逐日结算抵押物等来控制

流动性风险包括资产流动性风险和融资流动性风险,前者是指金融产品不能即使变现或由于市场效率低下而无法按正常的市场价格交易;后者是指银行的现金流不能及时满足支出的要求而导致银行违约或发生财物损失的可能性。融资风险可以通过恰当的现金流需求计划来控制,而现金流需求则可以通过对现金流缺口设置限制或进行多样化融资来加以控制。

操作风险通常被定义为由于人为的和技术的失误或意外事故所产生的风险,包括欺诈(交易对手故意伪造信息的情形)和管理失误以及不完善的程序和控制。防范操作风险的最好方法包括系统的备份、严格的内部控制规则与常规的应急计划。

法律风险通常是指交易对手或投资者在交易中蒙受损失进而决定起诉银行以免除责任时才出现,一般与信用风险相关,可以由银行的法规机构与高层管理人员来协调控制。

金融风险管理工具的研究发展历程:

\begin{tabular}{|l|l|}
\hline
年份  & 金融分析工具类型\\
\hline
1938  & 债券久期\\
1952 & 马科维茨(Markowitz)均值-方差模型\\
1964 & 夏普(Sharpe)资本资产定价模型\\
1972 & 默顿(Merton)连续跨期投资模型\\
1973 & 布莱克-斯科尔斯(Black-Scholes)期权定价模型\\
1974 & 默顿或有要求权定价模型\\
1979 & 二项式期权定价模型\\
1982 & 恩格尔(Engle)的ARCH模型\\
1983 & 风险调整资本收益率(RAROC)\\
1986 & 久期风险管理\\
1986 & 泰勒(Taylor)的SV随机波动模型\\
1987 & 互换风险管理\\
1988 & 银行风险加权资产\\
1992 & 压力测试\\
1993 & 在险价值VaR\\
1994 & 基于VaR的风险矩阵\\
1994 & 雅克奎尔(Jacquier)等基于贝叶斯原理的MCMC方法\\
1997 & 信用VaR矩阵\\
1998年至今 & 基于VaR的信用风险与市场风险方法\\
1999 & 阿茨纳(Artzner)等提出的相容性风险测度理念\\
2001 & M.德诺尔特(Denault)提出的资本配置公理化方法\\
2004 & 基于VaR的综合风险管理(包括操作风险)\\
\hline
\end{tabular}

目前,假设性情景相关规范可见于衍生性金融产品政策小组(DPG)对于“特定市场变动”的定义,以下8种情况作为压力情景的参考:
\begin{itemize}
\item 收益率曲线平移达100个基本点
\item 收益率曲线扭转达25个基本点
\item 3个月期的利率波动增加或减少20\%
\item 股价指数的变动达10\%
\item 主要国家汇率变动达6\%,其他货币汇率变动达20\%
\item 汇率波动增减达20\%
\item 互换契约利差达20个基本点
\end{itemize}

关于VaR计算方法的研究主要围绕历史模拟法、分析法(方差-协方差法)和蒙特卡洛模拟法三种方法展开的。

历史模拟法由于需要大量的样本以及金融资产的价格变动一般不满足同分布的假设,且两者之间存在不可调和的矛盾,正逐渐失去其应用价值。

学者认为,分析法在速度上优于蒙特卡洛方法,但准确性不如它。

大量的实证研究表明,在实际的金融市场上,大部分金融变量的标准差具有一些与正太假设不符的特征,如异方差和厚尾现象等。为了解决这些问题,恩格尔(1982)提出了ARCH和GARCH模型,并因此获得了2003年诺比尔经济学奖。

阿茨纳等人(1999)认为,风险是一个映射,如果一个风险计量方法在数理逻辑和经济逻辑上是合理的,则该风险应该满足相容性风险度量的四条公理性假设:
\begin{itemize}
\item 正齐次性Positive Homogeneity
\item 次可加性 Subadditivity
\item 单调性 Monotonicity
\item 转移不变性 Translation Invariance
\end{itemize}

目前,国际上具有代表性的基于VaR的信用风险管理模型:
\begin{itemize}
\item J.P. Morgan 1997年给出的Credit MetricsTM模型
\item CSFB 1997年给出的Credit Risk$^+$系统
\item Mckinsey 1998年给出的Credit Portfolio ViewTM系统
\end{itemize}

目前,国外关于操作风险的文献大都限于对新巴塞尔自本协议的介绍,对操作风险的信用量化分析极少,更谈不上相应的数据库建设。

\section{VaR的基本原理与应用分析}

金融经济学的实证研究表明:时间跨度越短,样本产生的数据越多,实际回报分布越接近于正态。

持有期越短月有利于满足组合头寸保持不变的假定。

VaR核心在于构造证券组合价值变化的概率分布。但VaR各种计算方法所依据的概率分布假设前提是:市场在刚刚过去的变化特征无偏地指示其在不远的将来的变化规律,这暗含市场比率(如利率、汇率、价格等市场因子)变化的概率分布具有稳定性并且是可以估计的。为了计算方便,尝尝假设在很短的时间内市场比率的变化服从(相互独立的)联合正态分布。因为近两个世纪以来,正态分布充分描述了许多客观现象的整体状况,一直在投资组合中起着核心作用。

VaR的计算方法大多是围绕着对资产报酬分布特征的确定而展开的,主要包括参数法、半参数法和非参数法。

\subsection{方差协方差方法(参数法)}
核心是基于资产报酬的方差——协方差矩阵进行估计。最有代表性的是目前流行的J.P. Morgan银行的Risk Metrics$^TM$方法。这种方法基于线性假定和正态假定,即假定持有期内资产价值的变化与其风险因素报酬成线性关系,风险因素报酬均服从正态分布。尽管这与实际情况不相吻合,仅限于线性的投资工具,金融资产收益率的分布与正态分布相比具有厚尾性,使金融风险的VaR值被低估,但由于它的可操作性和易于参数的敏感性分析,使其仍被广泛的使用。

如果收益率$R_i~N(\mu,\sigma^2)$,则VaR可表示成$$VaR=\mu+\sigma*\Phi^{-1}(Pr)$$这里,Pr是左尾概率,$\Phi(·)$是标准正态分布函数,参数$\mu、\sigma$可由极大似然估计得到。这种方法可由正态分布推广到其他累积概率分布函数,其中所有的不确定性都体现在$\sigma$上。

目前,该方法对线性假设做了不同程度的扩展,即资产的价格变化对于市场变量的变化可以是线性也可以是非线性的,因此又进一步分为组合正态法、资产正态法、Delta正态法和Delta-Gamma正态法。

\qd{组合正态法}不考虑市场变量的波动性,而是直接从整个投资组合的汇报来计算风险值,根据正态性假设,组合风险值可表示为:
$$VaR=z_\alpha \sigma_p \sqrt{\Delta t} \times V$$
其中$\sigma_p$是投资组合或业务部门的汇报标准差,V为头寸价值规模

\qd{资产正态法}对组合汇报标准差的计算不是直接来自组合汇报,而是
$$\sigma_p=\sqrt{w^T\sum w}$$
其中$w$为组合中个头寸的权重,$\sum$为各头寸回报直接的$n \times n$协方差矩阵。
这里假设各头寸回报的向量服从联合正态分布。

\qd{Delta正态法}是针对非线性类金融资产(如期权)来设计的。

\subsection{分析法(参数法)}

\subsection{厚尾模型Heavy Tail Model(半参数法)}

\subsection{估计函数模型Estimating Function Model(半参数法)}

\subsection{历史模拟法(非参数法)}

\subsection{蒙特卡洛模拟法(非参数法)}


\end{document}